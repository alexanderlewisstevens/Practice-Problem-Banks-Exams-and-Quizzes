\documentclass[12pt]{article}
\usepackage[margin=1in]{geometry}
\usepackage{titlesec}
\usepackage{enumitem}
\usepackage{parskip}

\title{Quiz 1 practice}
\date{}
\begin{document}



\section*{Quiz 1 Concepts}
\begin{itemize}
  \item \textbf{Brute Force Algorithms} practice solving the algorithms below using brute force, they don't need to be correct solutions, but they don't need to be efficient.
  \item Understand the basics of the three sorting algorithms:\newline
Be ready to apply the algorithms to an input\newline
Understand what input gives the best case and worst case for each algorithm\newline
Understand the procedures each algorithms uses and be ready to write pseudocode for them
\end{itemize}

\section*{Brute Force Problem Set}
Use any of the abstract data types (sets, maps, stacks etc.) that we have talked about.
\subsection*{Problem 1: Two Sum}
\textbf{Problem Statement:} Given an array of integers \texttt{nums} and an integer \texttt{target}, return the indices of the two numbers such that they add up to \texttt{target}. You may assume that each input would have exactly one solution, and you may not use the same element twice.

\textbf{Example:}
\begin{verbatim}
Input: nums = [2,7,11,15], target = 9
Output: [0,1]
\end{verbatim}


\vspace{1cm}

\subsection*{Problem 2: Best Time to Buy and Sell Stock}
\textbf{Problem Statement:} You are given an array \texttt{prices} where \texttt{prices[i]} is the price of a stock on the \texttt{i-th} day. Find the maximum profit from a single buy and sell operation. You must buy before you sell.

\textbf{Example:}
\begin{verbatim}
Input: prices = [7,1,5,3,6,4]
Output: 5 (buy at 1, sell at 6)
\end{verbatim}


\vspace{1cm}

\subsection*{Problem 3: Contains Duplicate}
\textbf{Problem Statement:} Given an array of integers, check if any value appears at least twice.

\textbf{Example:}
\begin{verbatim}
Input: [1, 2, 3, 1]
Output: True
\end{verbatim}


\vspace{1cm}

\subsection*{Problem 4: Maximum Subarray}
\textbf{Problem Statement:} Given an integer array \texttt{nums}, find the contiguous subarray (containing at least one number) with the largest sum.

\textbf{Example:}
\begin{verbatim}
Input: [-2,1,-3,4,-1,2,1,-5,4]
Output: 6 ([4,-1,2,1])
\end{verbatim}


\vspace{1cm}

\subsection*{Problem 5: Longest Common Prefix}
\textbf{Problem Statement:} Write a function to find the longest common prefix string amongst an array of strings. If there is no common prefix, return an empty string \texttt{""}.

\textbf{Example:}
\begin{verbatim}
Input: ["flower","flow","flight"]
Output: "fl"
\end{verbatim}


\vspace{1cm}

\subsection*{Problem 6: Valid Palindrome}
\textbf{Problem Statement:} Given a string, determine if it is a palindrome considering only alphanumeric characters and ignoring cases.

\textbf{Example:}
\begin{verbatim}
Input: "A man, a plan, a canal: Panama"
Output: True
\end{verbatim}


\vspace{1cm}

\subsection*{Problem 7: First Unique Character in a String}
\textbf{Problem Statement:} Given a string, find the first non-repeating character and return its index. If it does not exist, return -1.

\textbf{Example:}
\begin{verbatim}
Input: "leetcode"
Output: 0
\end{verbatim}


\vspace{1cm}

\subsection*{Problem 8: Intersection of Two Arrays}
\textbf{Problem Statement:} Given two arrays, return their intersection (unique elements only).

\textbf{Example:}
\begin{verbatim}
Input: nums1 = [1,2,2,1], nums2 = [2,2]
Output: [2]
\end{verbatim}


\vspace{1cm}

\end{document}
