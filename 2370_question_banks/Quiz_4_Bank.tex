\documentclass[12pt]{exam}

\usepackage{amsmath}
\usepackage{listings}
\usepackage{geometry}
\geometry{margin=1in}

\lstset{
  basicstyle=\ttfamily\footnotesize,
  frame=single,
  breaklines=true,
  postbreak=\mbox{\textcolor{red}{$\hookrightarrow$}\space}
}

\begin{document}

\title{Practice Problems: Quicksort, Convex Hull, and Recurrence Relations}
\date{}
\maketitle

\vspace{0.3in}
\noindent\textbf{Name:} \rule{4in}{0.4pt}

\vspace{0.3in}

\begin{questions}

% --------------------
% Quicksort
% --------------------

\section*{Quicksort and Randomized Quicksort}

\question[4] Describe how Quicksort partitions an array. What happens during each recursive step?

\vspace{1.2in}

\question[4] For the input array \([6, 5, 4, 1, 2, 3]\), trace the calls made by Quicksort when the last element is always chosen as pivot. Write the contents of the array after each 

\vspace{2in}

\question[4] What is the best-case scenario for Quicksort, and what input pattern produces it?

\vspace{1in}

\question[4] What is the worst-case scenario for Quicksort, and what input pattern produces it?

\vspace{1in}

\question[4] Write the recurrence for Quicksort's runtime in the best-case (perfectly balanced splits) and solve it using the Master Theorem.

\vspace{1.5in}

\question[4] Write the recurrence for Quicksort's runtime in the worst-case (one element per partition) and solve it.


% --------------------
% Convex Hull
% --------------------

\newpage
\section*{Convex Hull}

\question[4] Define the Convex Hull problem by stating what the input and output should represent.

\vspace{1.2in}

\question[4] Describe the brute-force approach to computing the convex hull of a set of 2D points. What is its runtime and why?

\vspace{1.5in}

\question[4] Describe the high-level idea behind the QuickHull algorithm.

\vspace{1.2in}

\question[4] What is the worst-case time complexity of QuickHull? Draw a point configurations where it occurs.

\vspace{1.2in}


\question[4] Describe the recursive structure of QuickHull and write its recurrence when splits are even.

\vspace{1.5in}


% --------------------
% Recurrences
% --------------------

\newpage
\section*{Writing and Solving Recurrences}

\question[4] Write a recurrence relation for the following pseudocode:

\begin{lstlisting}
def mystery(n):
    if n <= 1:
        return 1
    return 3 * mystery(n//2) + n
\end{lstlisting}

\vspace{1.2in}

\question[4] Solve the recurrence from the previous problem using the Master Theorem.

\vspace{1.2in}

\question[4] Use the recursion tree method to solve: \( T(n) = T(n/3) + n \)

\vspace{1.5in}

\question[4] Solve \( T(n) = 2T(n/2) + n \) using the Master Theorem.

\vspace{1.2in}

\question[4] Solve \( T(n) = T(n-1) + n \) using the recursion tree method.

\vspace{1.5in}

\question[4] Given a recurrence \( T(n) = 4T(n/2) + n^2 \), classify the function using the Master Theorem.

\vspace{1.2in}

\end{questions}

\end{document}